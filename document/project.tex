\documentclass[hidelinks,english]{article}

\usepackage{graphicx}
\usepackage{grffile}
\usepackage[T1]{fontenc}
\usepackage{babel}
\usepackage{hyperref}
\usepackage{array}
\usepackage{natbib}
\date{\today}

\graphicspath{{Pictures/}}
\begin{document}	
	\begin{titlepage}
		\pagenumbering{gobble}
		\begin{figure}[!t]
			\includegraphics[width=\linewidth]{up_logo.png}
		\end{figure}
		\vspace*{\stretch{1.5}}
		\begin{center}
			\huge{COS 326: Database Systems}\\
			\huge{Topic 5: NoSQL MongoDB vs SQL}\\
			\vspace{10mm}
		\end{center}
		\begin{center}
			\begin{tabular}{ c c c }
				Grobler, Arno & Harvey, Matthias & Lochner, Amy  \\
				\texttt{14011396} & \texttt{14027021} & \texttt{14038600} \\
				Madigoe, Pricilla & Maree, Armand & Obo, Diana \\
				\texttt{13049128} & \texttt{12017800} & \texttt{13134885} \\				
			\end{tabular}
		\end{center}
		\begin{center}
			Department of Computer Science, University of Pretoria
		\end{center}
		\vspace*{\stretch{2.0}}
	\end{titlepage}
	\newpage
	\tableofcontents
	\newpage
	\pagenumbering{arabic}
	
%	Start here with section 1
   \section{Introduction}
   % state the database problem area and solutions presented in the papers. State objectives of the essay
   
   
   \section{The Database Problem Area}
   %make subsections as you see fit
   \paragraph\indent
   In today's world the most common database implementation makes use of the relational model. However, due to the increase in the amount of data collected today and considering the fact that the data is often unstructured, an alternative database implementation must be considered, namely NoSQL. The problem is that while unstructured data is becoming more prominent there is still a need for modest-sized databases that hold structured data. This prompts the question on whether or not NoSQL databases will replace relational databases. As there is a general move towards NoSQL databases and the increasing availability of open-source NoSQL databases, it encourages the database designer to question whether he should make use of a NoSQL database. It is well known that NoSQL databases perform very well for large, unstructured collections of data but this begs the question of how a relational database will fair against a NoSQL database for a modest-sized database that holds structured data. This is what was researched. \citep{parker2013comparing}
   
   \section{Solution to the Problem}
   %make subsections as you see fit
   \paragraph\indent
   The NoSQL document and collection structure works similar to the row and table model of the relational database structure \cite{parker2013comparing} where each row has a key (unique ID) and some data item (value) \cite{upguardmysqlmongodb}. MongoDB stores most of these documents in memory which in turn speeds up the querying process significantly compared to the hard-drive querying process that MySQL uses \cite{parker2013comparing}. The reason for this is that since the data item is stored as a single document, the retrieval of these objects is faster than joining the values from separate tables \cite{upguardmysqlmongodb}. However, the NoSQL model does not guarantee a specific set of distinct values to be returned from a query, and this is where the SQL model comes in handy \cite{upguardmysqlmongodb}.
   
   \paragraph\indent
   There are a few factors that determine when a database will be more suited for a specific project. Firstly, since NoSQL databases do not conform to some scheme. If the initial requirements are not clear then NoSQL will adopt changes with no complaints \cite{bucklersqlnosql}. Secondly, since NoSQL prefers the denormalization of data it is significantly faster when it comes to queries, but at the cost of update speed \cite{bucklersqlnosql}. And lastly, NoSQL scales a lot better than SQL which will simplify the process of distributed databases - should a medium sized business need to expand \cite{upguardmysqlmongodb}.
   
   \section{Advantages \& Disadvantages of the proposed solutions}
   %make subsections as you see fit
\subsection{Advantages}
MongoDB improves scalability \cite{parker2013comparing} by scaling well horizontally i.e. scale across several servers  \cite{upguardmysqlmongodb}. The data field and the value for that field is stored together as one record  \cite{upguardmysqlmongodb} which makes data retrieval much faster. MongoDB gains its performance by key value design, ease to scale out and denormalization  \cite{MakbleAdvanDisadvan}. It has better runtime performance for inserts, updates and deletes \cite{parker2013comparing}. MongoDB has a flexible schema \cite{MakbleAdvanDisadvan}, which allows easy manipulation with the JSON format. \\[0.5cm]
The tight rules that govern a SQL DB structure ensures data integrity and security without having to rely on application rules and logic \cite{upguardmysqlmongodb}.  It is a simple way of representing data or business logic \cite{upguardmysqlmongodb} because of its structured schema. SQL DB has an easy-to-use language (SQL) to retrieve and query data \cite{upguardmysqlmongodb}. It performs better when updating and querying  non-key attributes \cite{parker2013comparing, upguardmysqlmongodb}. 

\subsection{Disadvantages:}
MongoDb performs poorly for aggregate functions and querying based on non-key values \cite{parker2013comparing}. It has memory limitations because the size of the DB is limited by virtual memory provided by OS and hardware \cite{MakbleAdvanDisadvan}. MongoDB has no built in way to retrieve an object based on reference \cite{parker2013comparing}.\\[0.5cm]
SQL DB requires additional joins in more complex schemas \cite{parker2013comparing, upguardmysqlmongodb} which leads to high transaction loads and decreased performance. It does not scale well across several servers \cite{upguardmysqlmongodb} i.e. poor horizontal scalability.

   
   \section{Relevance to COS326 and the business industry}
    %make subsections as you see fit
    
    \section{Conclusion}
%	bibliograpy starts here. Keep this at the end of the page
	\bibliographystyle{plain}
	\bibliography{bibfile}
	
\end{document}